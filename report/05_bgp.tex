\section{BGP routing table}
\label{sec:bgp}


\subsection{Announced IP block sizes}

\subsection{Age distribution of BGP entries}
Finally, eighth, we will provide a cumulative distribution function (CDF) of prefix ages, by which to draw conclusions about the stability of the routing table content. These will give information not so much about what is happening, but who is involved and for how long.

\subsection{BGP announcements by geographical region}
One other direction of our BGP routing system study deals with characteristics of the allocations and announcements themselves.  These are in the areas of geography and age.  Some of the new areas we will examine to monitor where allocation activity is happening, and for how long a time on average.  Thus, seventh, we will give a year-on-year comparison of prefix allocations and prefix announcements by major countries, including Russia, Japan, major countries of Asia, and broad regions such as the European Union, North America, and Africa.

\subsection{BGP table growth accelerators}

\subsubsection{IP block fragmentation}
Third, we will build a graph showing both the approximated BGP table size and a calculated table size had there been no fragmenting of allocated blocks.

\subsubsection{Duplicate announcements of IP blocks}
Fifth, we will determine the trends of the percentage of covering prefixes that 1) match, 2) fragment, and 3) aggregate allocations.
Sixth, we will show the dynamics of "Level 1" and "Level 2+" covered prefixes over time. Plus, we will give a comparison of covering prefixes to the number of allocations.  All these measurements concern the overall growth of the BGP routing table.