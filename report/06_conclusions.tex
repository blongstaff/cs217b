\section{Conclusions}
\label{sec:conclusions}

We have thoroughly analyzed BGP announcement dumps available from University
of Oregon RouteViews and RIPE NCC Routing Information Service projects. The
general conclusion of our analysis is that average size of the BGP routing
table increased more than 2 times in last 6 years. The IP address allocations
are also doubled in the same period, but numerically, all new allocated blocks
accounts less than 18\% of the actual entries in the BGP routing table. The
primary causes of the accelerated BGP routing table growth are allocated IP
block fragmentation -- more than 80\% of announced prefixes are parts of
allocated IP blocks; and announced space duplication -- more than 54\% of
address space is covered at least twice in the global routing table. This
highlights an emerged problem of a trend to use the global routing table to
serve local interests, e.g., implementing traffic engineering and multihoming
techniques.

The content analysis of the BGP announcements shows that majority of the
globally announced prefixes ($>$50\%) have size /24. This additionally
strengthen the conclusion that most entries in global routing table serve not
a global, but local interests of small customer networks. Additionally, we
conclude that the global routing table is very dynamic. Although, there is a
small portion of highly stable entries ($<$15\%), the rest of the BGP table
content shows the exponential distribution of prefix longevity.

Analysis of geographical distribution of the IP allocation and BGP announced
prefixes shows the varied penetration of Internet in a global scale. The
interesting fact that geographical distribution of the number of allocated
prefixes as well as numbers of corresponding address space, number of
announced prefixes, and corresponding globally announced address space
displays a quasi-exponential distribution. Moreover, this distribution has not
been changing it's character during last 6 years. Another findings is that
different countries has different IP address space utilization patterns. For
example, in Japan tend to announce smaller prefixes (i.e., more addresses
covered) than South Korea. Numerically, one announced prefix belonging to
Japan covers 37,200 IP addresses, while in South Korea this number is only
5,900. If this difference happens because of additional government
regulations, than for future IPv6 deployment, we should consider an
establishment of similar global regulations.
